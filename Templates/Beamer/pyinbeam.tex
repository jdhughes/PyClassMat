
\documentclass{beamer}
% \documentclass[handout,xcolor=pdftex,dvipsnames,table]{beamer}

\mode<presentation>
{
  \usetheme{Warsaw}
  \usecolortheme{whale}
  % or ...

  \setbeamercovered{transparent}
  % or whatever (possibly just delete it)
  \setbeamertemplate{navigation symbols}{}
}

\setbeamertemplate{itemize items}[ball]
\setbeamertemplate{itemize subitem}[triangle]
\setbeamertemplate{itemize subsubitem}[circle]
\setbeamercovered{transparent}

\usepackage{palatino} 
\usepackage{listings} % Gives syntax highlighting for python code. 
\usepackage{color} % Used for syntax highlighting. 
\usepackage{textcomp} % Used for syntax highlighting. 

% This gives syntax highlighting in the python environment 
\definecolor{gray}{gray}{0.5} 
\definecolor{key}{rgb}{0,0.5,0} 
\lstset{
language=python,
basicstyle=\ttfamily\tiny, 
otherkeywords={1, 2, 3, 4, 5, 6, 7, 8 ,9 , 0, -, =, +, [, ], (, ), \{, \}, :, *, !}, 
keywordstyle=\color{blue}, 
stringstyle=\color{red},
showstringspaces=false,
alsoletter={1234567890},
otherkeywords={\ , \}, \{},
keywordstyle=\color{blue},
emph={access,and,break,class,continue,def,del,elif ,else,%
except,exec,finally,for,from,global,if,import,in,i s,%
lambda,not,or,pass,print,raise,return,try,while},
emphstyle=\color{black}\bfseries,
emph={[2]True, False, None, self},
emphstyle=[2]\color{green},
emph={[3]from, import, as},
emphstyle=[3]\color{blue},
upquote=true,
morecomment=[s]{"""}{"""},
commentstyle=\color{gray}\slshape,
emph={[4]1, 2, 3, 4, 5, 6, 7, 8, 9, 0},
emphstyle=[4]\color{blue},
literate=*{:}{{\textcolor{blue}:}}{1}%
{=}{{\textcolor{blue}=}}{1}%
{-}{{\textcolor{blue}-}}{1}%
{+}{{\textcolor{blue}+}}{1}%
{*}{{\textcolor{blue}*}}{1}%
{!}{{\textcolor{blue}!}}{1}%
{(}{{\textcolor{blue}(}}{1}%
{)}{{\textcolor{blue})}}{1}%
{[}{{\textcolor{blue}[}}{1}%
{]}{{\textcolor{blue}]}}{1}%
{<}{{\textcolor{blue}<}}{1}%
{>}{{\textcolor{blue}>}}{1},%
numbers=left,
}



\begin{document}


\begin{frame}[fragile]{Code In Frame}
Here is some python code.  Note that the [fragile] keyword is required on the begin frame line.  See the following site for additional details: \url{http://robfelty.com/2008/09/22/beamer-fragile-frames}
\begin{lstlisting}
class GridNodeIterator(object):
    def __init__(self, grid):
        self.index = 0
        return
        
    def __iter__(self):
        return self
        
    def next(self):
        if self.index == self.nodes - 1:
            raise StopIteration()
        nodeobj = self.get_nodeobj(self.index)
        self.index += 1
        return nodeobj$
\end{lstlisting}
\end{frame}


\begin{frame}{Code In Figure In Frame}
Here is some python code.  In this case, the code was taken out of a python source file.
  \begin{figure}
  \centering
        \lstset{numbers=none}
        \lstinputlisting[language=python, firstline=9,lastline=20]{pyinbeam.py}
   \end{figure}

\end{frame}

\begin{frame}{Code In Figure In Frame}
Here is some python code.  In this case, the code was taken out of a python source file.
  \begin{figure}
  \centering
        \lstset{numbers=none}
        \lstinputlisting[language=python, firstline=21,lastline=43]{pyinbeam.py}
   \end{figure}

\end{frame}



\end{document}


