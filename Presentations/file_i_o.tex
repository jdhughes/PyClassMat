\documentclass{beamer}
% \documentclass[handout,xcolor=pdftex,dvipsnames,table]{beamer}

\mode<presentation>
{
  \usetheme{Warsaw}
  \usecolortheme{whale}
  % or ...

  % or whatever (possibly just delete it)
  \setbeamertemplate{navigation symbols}{}
}


% make a command to decrease font size locally when desired
\newcommand\Fontvi{\fontsize{8.5}{7.2}\selectfont}



\setbeamertemplate{itemize items}[ball]
\setbeamertemplate{itemize subitem}[triangle]
\setbeamertemplate{itemize subsubitem}[circle]
\setbeamercovered{invisible}

\usepackage{palatino} 
\usepackage{listings} % Gives syntax highlighting for python code. 
\usepackage{color} % Used for syntax highlighting. 
\usepackage{textcomp} % Used for syntax highlighting. 
\usepackage{caption}
\captionsetup{labelformat=empty,labelsep=none}
% This gives syntax highlighting in the python environment 
\definecolor{gray}{gray}{0.5} 
\definecolor{key}{rgb}{0,0.5,0} 
\lstset{
language=python,
basicstyle=\ttfamily\tiny, 
otherkeywords={1, 2, 3, 4, 5, 6, 7, 8 ,9 , 0, -, =, +, [, ], (, ), \{, \}, :, *, !}, 
keywordstyle=\color{blue}, 
stringstyle=\color{red},
showstringspaces=false,
alsoletter={1234567890},
otherkeywords={\ , \}, \{},
keywordstyle=\color{blue},
emph={access,and,break,class,continue,def,del,elif ,else,%
except,exec,finally,for,from,global,if,import,in,is,%
lambda,not,or,pass,print,raise,return,try,while},
emphstyle=\color{black}\bfseries,
emph={[2]True, False, None, self},
emphstyle=[2]\color{green},
emph={[3]from, import, as},
emphstyle=[3]\color{blue},
upquote=true,
morecomment=[s]{"""}{"""},
commentstyle=\color{gray}\slshape,
emph={[4]1, 2, 3, 4, 5, 6, 7, 8, 9, 0},
emphstyle=[4]\color{blue},
literate=*{:}{{\textcolor{blue}:}}{1}%
{=}{{\textcolor{blue}=}}{1}%
{-}{{\textcolor{blue}-}}{1}%
{+}{{\textcolor{blue}+}}{1}%
{*}{{\textcolor{blue}*}}{1}%
{!}{{\textcolor{blue}!}}{1}%
{(}{{\textcolor{blue}(}}{1}%
{)}{{\textcolor{blue})}}{1}%
{[}{{\textcolor{blue}[}}{1}%
{]}{{\textcolor{blue}]}}{1}%
{<}{{\textcolor{blue}<}}{1}%
{>}{{\textcolor{blue}>}}{1},%
numbers=none,
}

\newcommand{\putat}[3]{\begin{picture}(0,0)(0,0)\put(#1,#2){#3}\end{picture}}



\title[]{Python Workshop\\
File Input/Output}

\author[Fienen] % (optional, use only with lots of authors)
{Mike~Fienen}
\institute[USGS] % (optional, but mostly needed)
{
  U.S. Geological Survey\\
  Wisconsin Water Science Center, Middleton, Wisconsin USA
  }
  \titlegraphic{\includegraphics[scale=0.5]{figures/c_USGSid1.pdf}}
  

\date[UQ12] % (optional, should be abbreviation of conference name)
{USGS National Groundwater Workshop, August 2012}

\subject{Python}


\begin{document}

\begin{frame}
  \titlepage
\end{frame}
\logo{\vspace{-0.3cm} \includegraphics[width=1.0cm]{figures/c_USGSid1.pdf}\hspace*{11.10cm}}  

\begin{frame}{Outline}
\tableofcontents
\end{frame}

\begin{frame}[fragile]
\frametitle{Overview}
\begin{itemize}

\item Much of what is useful to do in Python is reading files, manipulating the data, and writing out results in another format
\item Python and Numpy provide ways to read and write ASCII and binary files
\item We will focus on ASCII files
\end{itemize}
\end{frame}


\section{Reading and Writing Strings}

\begin{frame}[fragile]
\frametitle{Reading and Writing with Strings}
\begin{itemize}
\item The simplest way to write information to a string is using \texttt{str}
\begin{lstlisting}
>>>a = 5.4
>>>str(a)
'5.4'
\end{lstlisting}
\item We typically want more control. Two main ways to do it. Old school (\texttt{\%}) and new school (\texttt{format})
\item Formatted input and output are a key difference between Python 2.X and 3.X

\end{itemize}
\end{frame}


\begin{frame}[fragile]
\frametitle{Writing Strings the Old School Way (\texttt{\%})}

\begin{itemize}
\begin{small}
\item The general syntax is to make a string with conversion types for variables. For example:
\begin{lstlisting}
>>>outstr = 'I have %21.1f kg of %s and %d bins of %s' %(3.99983,'eggs',53,'spam')
>>> outstr
'I have                   4.0 kg of eggs and 53 bins of spam'
>>>outstr = 'I have %-21.1f kg of %s and %0.3d bins of %s' %(3.99983,'eggs',53,'spam')
>>> outstr
'I have 4.0                   kg of eggs and 053 bins of spam'

\end{lstlisting}
\pause
\item The general idea is to make a string including \texttt{'\%'}, a conversion flag (optional), a width and resolution (optional), and a conversion type (required).
\item[] For example: \newline{}
\texttt{\%<flag><width>.<resolution><type>}\newline{}\texttt{\%-12.3f} Is a left-justified, floating point value with width of 12 and 3 decimal places.
\pause
\item Following the format string must be a list of values as a tuple identified by \texttt{\%}

\item[] \texttt{'\%4d is the \%s\textbackslash{}n' \%(42,'answer')}
\end{small}
\end{itemize}

\end{frame}



\begin{frame}[fragile]
\frametitle{Writing Strings the Old School Way (\texttt{\%})}
Details about formatted output available at:\newline{}
\begin{small} \url{http://docs.python.org/library/stdtypes.html} \end{small}

\begin{itemize}
\item Conversion flag characters
\begin{description}
\Fontvi
\item [\texttt{'\#'}] Invokes alternate behavior (see website for details)
\item [\texttt{'0'}] Pads numeric values with zeros
\item [\texttt{'-'}] Left-adjusts the output
\item [\texttt{' '}] Leave a space before signed positive values so they line up with negative ones

\end{description}
\pause
\item Most common conversion types. 
\end{itemize}
\begin{description}
\Fontvi
\item [\texttt{\%d} or \texttt{\%i} ] Signed integer
\item [\texttt{\%f} or \texttt{\%F} ] Floating point
\item [\texttt{\%e} or \texttt{\%E} ] Floating point exponential (lower or upper case)
\item [\texttt{\%g} or \texttt{\%G} ] Combination of \texttt{\%f} and \texttt{\%e} depending on resolution
\item [\texttt{\%s} or \texttt{\%r} ] String. Width is used, but not resolution

\end{description}
\end{frame}

\begin{frame}[fragile]
\frametitle{Writing Strings the Old School Way (\texttt{\%})}
Try it out!
\texttt{'\%4d is the \%s\textbackslash{}n' \%(42,'answer')}\newline{}\texttt{\%<flag><width>.<resolution><type>}
\begin{itemize}
\item Conversion flag characters
\begin{description}
\Fontvi
\item [\texttt{'\#'}] Invokes alternate behavior (see website for details)
\item [\texttt{'0'}] Pads numeric values with zeros
\item [\texttt{'-'}] Left-adjusts the output
\item [\texttt{' '}] Leave a space before signed positive values so they line up with negative ones

\end{description}
\item Most common conversion types. 
\end{itemize}
\begin{description}
\Fontvi
\item [\texttt{\%d} or \texttt{\%i} ] Signed integer
\item [\texttt{\%f} or \texttt{\%F} ] Floating point
\item [\texttt{\%e} or \texttt{\%E} ] Floating point exponential (lower or upper case)
\item [\texttt{\%g} or \texttt{\%G} ] Combination of \texttt{\%f} and \texttt{\%e} depending on resolution
\item [\texttt{\%s} or \texttt{\%r} ] String. Width is used, but not resolution

\end{description}
\end{frame}



\begin{frame}[fragile]
\frametitle{Writing Strings the New School Way (\texttt{format})}
Details about new school string formatting at:\newline{}
\begin{small}
\Fontvi
\url{http://docs.python.org/library/string.html#formatstrings}
\end{small}
\begin{itemize}
\item The general syntax is similar, but conversion information is supplied differently. For example:
\begin{lstlisting}
>>>outstr = 'I have %{0:21.1f} kg of {1:s} and {2:0=3d} bins of {3:s}'.
							format(3.99983,'eggs',53,'spam')
>>> outstr
'I have                   4.0 kg of eggs and 053 bins of spam'
\end{lstlisting}
\item In this case, make a string including \texttt{\{...\}}, statements with conversion information.

The general pattern is \texttt{\{[index]:[format]\}}\newline{}
\begin{itemize}
\item The \texttt{[index]} argument refers to the item index being mapped in\newline{}
\item The \texttt{[format]} argument is similar to those in the old school way, but with some additional flexibility
\end{itemize}

\end{itemize}
\end{frame}
\section{Text File Reading and Writing}
\begin{frame}[fragile]
Now that we can write strings, we can write them to files \newline{} \\
But first, We need to read stuff back in \emph{from} files
\end{frame}

\begin{frame}[fragile]
\frametitle{Interacting with Text Files}
\begin{itemize}
\item The first thing is to open a file and make a file object
\begin{lstlisting}
ifp = open('somefile.txt','r')
ofp = open('someotherfile.txt','w')
\end{lstlisting}
\begin{itemize}
\item This object can be used to read or write from. \\I use \texttt{ifp} for ``input file pointer" and\\ \texttt{ofp} for ``output file pointer" \\
The arguments \texttt{'r'} and \texttt{'w'} indicate ``read" and ``write" respectively.
\end{itemize}
\item To read the file can use \texttt{readline()} or \texttt{readlines()}\\
\begin{itemize}
\item The difference is that  \texttt{readlines()} reads the entire file into memory rather than  \texttt{readline()}  which reads one line at a time. Most of the time,  \texttt{readlines()}  is better
\item With \texttt{readlines()}  once the data are read in, the result is a list with each element representing a line in the text file
\end{itemize}
\end{itemize}
\end{frame}

\begin{frame}[fragile]
\frametitle{\texttt{Readlines} Example}
\begin{tiny}
Here's a file called \texttt{example\_file.txt}
\begin{verbatim}
This is a file.
It is a very nice file, no?
It has several lines.
Some of them are numbers.
This is NOT a poem.
3.14 pi 5
But code --> now THAT's poetry!
\end{verbatim}

\pause
Now let's read it in:
\begin{lstlisting}
>>>import os
>>>infilepath = os.path.join('data','example_file.txt')
>>>whole_file = open(infilepath,'r').readlines()
>>> whole_file
['This is a file.\n', 'It is a very nice file, no?\n', 'It has several lines.\n', 
'Some of them are numbers.\n', 'This is NOT a poem.\n', '3.14 pi 5\n',
"But code --> now THAT's poetry!"]
\end{lstlisting}
\pause
What if it's a windows file?
\begin{lstlisting}
>>>infilepath = os.path.join('data','example_file_win.txt')
>>>whole_file = open(infilepath,'r').readlines()
>>> whole_file
['This is a file.\r\n', 'It is a very nice file, no?\r\n', 'It has several lines.\r\n', 
'Some of them are numbers.\r\n', 'This is NOT a poem.\r\n', '3.14 pi 5\r\n',
"But code --> now THAT's poetry!"]
\end{lstlisting}
\end{tiny}



\end{frame}

\begin{frame}[fragile]
\frametitle{Parsing input strings}
\begin{itemize}
\item Using \texttt{strip} and \texttt{split}
\begin{itemize} 
\item \texttt{strip()} removes newline and tab characters from the end 

\begin{lstlisting}
>>>line = 'USGS        430406089232901 2010-12-03      15.04   P\t\r\n'
>>>line.strip()
'USGS\t430406089232901\t2010-12-03\t15.04\tP'
\end{lstlisting}
\end{itemize}
\pause
\begin{itemize} 
\item \texttt{split()} breaks up a string on whitespace 
\item Can take any character as an argument (usually ',' or '~') 
\begin{lstlisting}
>>>line.strip().split()
['USGS', '430406089232901', '2010-12-03', '15.04', 'P']
>>>line.strip().split('0')
['USGS\t43', '4', '6', '892329', '1\t2', '1', '-12-', '3\t15.', '4\tP']
\end{lstlisting}
\pause
\item Stacking \texttt{strip} and \texttt{split} is a common violation of the general rule not to stack up function calls
\end{itemize}
\end{itemize}
\end{frame}

\begin{frame}[fragile]
\frametitle{Parsing input strings (continued)}
\begin{itemize}
\item Using \texttt{pop}
\begin{lstlisting}
>>>a=line.strip().split()
>>>a
['USGS', '430406089232901', '2010-12-03', '15.04', 'P']
>>>a.pop()
'P'
>>>a
['USGS', '430406089232901', '2010-12-03', '15.04']
>>>a.pop(1)
'430406089232901'
>>>a
['USGS', '2010-12-03', '15.04']
\end{lstlisting}
\begin{itemize} 
\item \texttt{pop()} \emph{both} returns an element from a list \emph{and} removes it from the remaining list
\end{itemize}
\pause
\item Regular expressions are very flexible but another topic 
\begin{lstlisting}
>>>import re
>>>allints = re.findall("[0-9]",line)
>>>allints
['4', '3', '0', '4', '0', '6', '0', '8', '9', '2', '3', '2', '9', '0', '1', '2', 
'0', '1', '0', '1', '2', '0', '3', '1', '5', '0', '4']

\end{lstlisting}

\end{itemize}
\end{frame}

\begin{frame}[fragile]
\frametitle{An Example Text File from NWIS}
\begin{tiny}
\begin{verbatim}
# ---------------------------------- WARNING ----------------------------------------
# Provisional data are subject to revision. Go to
# http://waterdata.usgs.gov/nwis/help/?provisional for more information.
#
# File-format description:  http://waterdata.usgs.gov/nwis/?tab_delimited_format_info
# Automated-retrieval info: http://waterdata.usgs.gov/nwis/?automated_retrieval_info
#
# Contact:   gs-w_support_nwisweb@usgs.gov
# retrieved: 2012-07-16 17:24:35 EDT	(vaas01)
#
# Data for the following 2 site(s) are contained in this file
#    USGS 430406089232901 DN-07/09E/23-1297
#    USGS 430427089284901 DN-07/09E/19-0064
# -----------------------------------------------------------------------------------
#
# Data provided for site 430406089232901
#    DD parameter statistic   Description
#    01   72019     00001     Depth to water level, feet below land surface (Maximum)
#
# Data-value qualification codes included in this output:
#     P  Provisional data subject to revision.
#
agency_cd	site_no	datetime	01_72019_00001	01_72019_00001_cd
5s	15s	20d	14n	10s
USGS	430406089232901	2010-12-03	15.04	P
USGS	430406089232901	2010-12-04	14.92	P
...
\end{verbatim}
\end{tiny}
\end{frame}


\begin{frame}[fragile]
\frametitle{Reading NWIS Output File}
\begin{lstlisting}
def NWIS_reader(infile):
    '''
    NWIS_reader(infile)
    A function to read in an NWIS file generated using USGS webservices.
    Mike Fienen - 7/16/2012
    <mnfienen *at* usgs *dot* gov>
    
    INPUT:
    infile --> the name of an input file in USGS RDB (tab-delimited) format
    
    OUTPUT:
    indat --> a dictionary with keys corresponding to site numbers and each
            element being a dictionary with keys date and depth to water.
    '''
    # tell the user what's happening
    print 'Reading Data from file: %s' %(infile)
    #
    # set up a couple initial variables
    #
    # format for reading the date --> formats noted at 
    #        http://docs.python.org/library/datetime.html (bottom of the page)
    indatefmt = "%Y-%m-%d"
    
\end{lstlisting}
\end{frame}
    
\begin{frame}[fragile]
\frametitle{Reading NWIS Output File (Continued)}
\begin{lstlisting}
    # open the text file and read all lines into a variable "tmpdat"
    tmpdat = open(infile,'r').readlines()
    
    # make empty lists to temporarily hold the data
    Site_ID = []   # site ID
    dates = []     # date of measurement (daily values)
    DTW = []       # depth to water below land surface (feet)
    prov_code = [] # provisional code: [P] is provisional, [A] is accepted
    # loop over the input data, keep only proper data rows. Parse and assign to lists
    for lnum, line in enumerate(tmpdat):
        # first read the lookup information from the header of the file
        if ("data for the following" in line.lower()):
            nWells = int(re.findall("[0-9]+",line)[0])
            statnums = []
            countynums = []
            for cwell in np.arange(nWells):
                nextline = lnum+1+cwell
                tmp = tmpdat[nextline].strip().split()
                statnums.append(tmp[2])
                countynums.append(tmp[3])
            station_lookup = dict(zip(statnums,countynums))
                
        if (('usgs' in line.lower()) and ('#' not in line)):
            tmp = line.strip().split() # strip newline off the end and split on whitespace
            Site_ID.append(tmp[1])
            dates.append(datetime.strptime(tmp[2],indatefmt)) #convert date to a time tuple
            DTW.append(tmp[3])
            prov_code.append(tmp[4].lower()) # --> note conversion to lower case!
\end{lstlisting}
\end{frame}

\begin{frame}[fragile]
\frametitle{Try Running the Code}
\begin{small}
Quick aside on \texttt{os} module...
\newline{}Windows uses \texttt{"\textbackslash"} for separating path components, but everyone else uses \texttt{"/"} \\
\begin{columns}[c]
\column{.4\textwidth}

So, you can make a Windows path as:
\begin{lstlisting}
>>>infile = '..\\data\\NWIS_data.dat'
\end{lstlisting}
\pause
\column{0.55\textwidth}
That works, but only on Windows. Better style is:
\begin{lstlisting}
>>>infile = os.path.join('..','data','NWIS_data.dat')
>>>infile # on a Mac
'../data/NWIS_data.dat'
>>>infile # on Windows
'..\\data\\NWIS_data.dat'

\end{lstlisting}
\end{columns}
\begin{center}\emph{That is platform independent}\end{center}
\pause
Now, let's try an example in the EXERCISES/FILE\_I\_O directory
\begin{lstlisting}
>>> import NWIS_read_parse as NWR, os
>>> infile = os.path.join('..','data','dane_county_GW_wells.dat')
>>>indat,county_lookup = NWR.NWIS_reader(infile)
Reading Data from file: ../data/dane_county_GW_wells.dat
Reading ../data/dane_county_GW_wells.dat complete!
>>>indat.keys()
>>>county_lookup.keys()
\end{lstlisting}
\end{small}
\end{frame}

\begin{frame}[fragile]
\frametitle{\texttt{np.genfromtxt}: flexible way to read columns}

Example file: \texttt{STATE\_FIPS.csv} 
\begin{tiny}
\begin{verbatim}
State Abbreviation,FIPS Code,State Name
AK,02,ALASKA
AL,01,ALABAMA
AR,05,ARKANSAS
AS,60,AMERICAN SAMOA
AZ,04,ARIZONA
CA,06,CALIFORNIA
...
\end{verbatim}
\end{tiny}
\pause
\begin{lstlisting}
import numpy as np
infilename = 'STATE_FIPS.csv'
indat = np.genfromtxt(infilename,delimiter=',',dtype=None,names=True)
\end{lstlisting}
\pause
\begin{description}
\begin{tiny}
\item[\texttt{delimiter=','}] delimiter can be \emph{anything}
\item[\texttt{dtype=None}] Numpy interprets column data types. If unknown, makes it a string
\item[\texttt{names=True}] Each column gets a data type and a name
\end{tiny}
\end{description}
\pause
\begin{lstlisting}
In [10]: indat
Out[7]: 
array([('AK', 2, 'ALASKA'), ('AL', 1, 'ALABAMA'), ('AR', 5, 'ARKANSAS'), ... 
dtype=[('State_Abbreviation', '|S2'), ('FIPS_Code', '<i4'), ('State_Name', '|S20')])
\end{lstlisting}
\pause
\hskip 1.5 cm \emph{N.B.$\rightarrow$underscore replaces space in names}
\end{frame}

\begin{frame}[fragile]
\frametitle{Writing Back out to a File}
\begin{itemize}
\item First need a file object in the same way as reading\\
\begin{lstlisting}
ofp = open('some_outfile.txt','w')
\end{lstlisting}
\item Next, create a string of output
\item Write the string using 
\begin{lstlisting}
ofp.write(<string>)
\end{lstlisting}
\item Remember to put a newline character \texttt{'\textbackslash n'}~at the end of each line 

\end{itemize}
\end{frame}

\section{Data from the Web}
\begin{frame}[fragile]
\frametitle{Pulling a data file from the Web}
\begin{itemize}

\item An example using REST (Representational State Transfer a.k.a. a RESTful query) of USGS water data.
\begin{lstlisting}
import urllib
fullURL = 'www.place.gov/some_path_to_a_data_file.txt'
datastream = urllib.urlopen(fullURL).read()
outfilename = 'local_filename.txt'
open(outfilename,'wb').write(datastream)
\end{lstlisting}

\item \texttt{urllib} enables simple interaction with a URL
\item BeautifulSoup allows for much more sophisticated complete web-scraping applications (not built in)
\item Writing the local version of the file as binary is most robust:
\begin{itemize}
\item Retains a copy of exactly what was downloaded
\item Writes the copy without respect to formatting issues
\end{itemize}
\item One could work only in memory and not make a local copy
\end{itemize}
\end{frame}

\begin{frame}[fragile]
\frametitle{Some Useful Resources}
\begin{itemize}

\item Building queries for RESTful queries of USGS water data \\  \url{http://waterservices.usgs.gov/}
\item USGS Water data type pm code lookup \\
\url{http://nwis.waterdata.usgs.gov/nwis/pmcodes/}
\item General I/O information in Python documentation \\ 
\url{http://docs.python.org/tutorial/inputoutput.html}
\end{itemize}
\end{frame}

\end{document}