\documentclass{beamer}
% \documentclass[handout,xcolor=pdftex,dvipsnames,table]{beamer}

\mode<presentation>
{
  \usetheme{Warsaw}
  \usecolortheme{whale}
  % or ...

  % or whatever (possibly just delete it)
  \setbeamertemplate{navigation symbols}{}
}

\setbeamertemplate{itemize items}[ball]
\setbeamertemplate{itemize subitem}[triangle]
\setbeamertemplate{itemize subsubitem}[circle]
\setbeamercovered{invisible}

\usepackage{palatino} 
\usepackage{listings} % Gives syntax highlighting for python code. 
\usepackage{color} % Used for syntax highlighting. 
\usepackage{textcomp} % Used for syntax highlighting. 
\usepackage{caption}
\captionsetup{labelformat=empty,labelsep=none}
% This gives syntax highlighting in the python environment 
\definecolor{gray}{gray}{0.5} 
\definecolor{key}{rgb}{0,0.5,0} 
\lstset{
language=python,
basicstyle=\ttfamily\tiny, 
otherkeywords={1, 2, 3, 4, 5, 6, 7, 8 ,9 , 0, -, =, +, [, ], (, ), \{, \}, :, *, !}, 
keywordstyle=\color{blue}, 
stringstyle=\color{red},
showstringspaces=false,
alsoletter={1234567890},
otherkeywords={\ , \}, \{},
keywordstyle=\color{blue},
emph={access,and,break,class,continue,def,del,elif ,else,%
except,exec,finally,for,from,global,if,import,in,is,%
lambda,not,or,pass,print,raise,return,try,while},
emphstyle=\color{black}\bfseries,
emph={[2]True, False, None, self},
emphstyle=[2]\color{green},
emph={[3]from, import, as},
emphstyle=[3]\color{blue},
upquote=true,
morecomment=[s]{"""}{"""},
commentstyle=\color{gray}\slshape,
emph={[4]1, 2, 3, 4, 5, 6, 7, 8, 9, 0},
emphstyle=[4]\color{blue},
literate=*{:}{{\textcolor{blue}:}}{1}%
{=}{{\textcolor{blue}=}}{1}%
{-}{{\textcolor{blue}-}}{1}%
{+}{{\textcolor{blue}+}}{1}%
{*}{{\textcolor{blue}*}}{1}%
{!}{{\textcolor{blue}!}}{1}%
{(}{{\textcolor{blue}(}}{1}%
{)}{{\textcolor{blue})}}{1}%
{[}{{\textcolor{blue}[}}{1}%
{]}{{\textcolor{blue}]}}{1}%
{<}{{\textcolor{blue}<}}{1}%
{>}{{\textcolor{blue}>}}{1},%
numbers=none,
}

\newcommand{\putat}[3]{\begin{picture}(0,0)(0,0)\put(#1,#2){#3}\end{picture}}



\title[]{Python Workshop\\
File Input/Output}

\author[Fienen] % (optional, use only with lots of authors)
{Mike~Fienen}
\institute[USGS] % (optional, but mostly needed)
{
  U.S. Geological Survey\\
  Wisconsin Water Science Center, Middleton, Wisconsin USA
  }
  \titlegraphic{\includegraphics[scale=0.5]{figures/c_USGSid1.pdf}}
  

\date[UQ12] % (optional, should be abbreviation of conference name)
{USGS National Groundwater Workshop, August 2012}

\subject{Python}


\begin{document}

\begin{frame}
  \titlepage
\end{frame}
\logo{\vspace{-0.3cm} \includegraphics[width=1.5cm]{figures/c_USGSid1.pdf}\hspace*{11.10cm}}  

\begin{frame}{Outline}
\tableofcontents
\end{frame}

\begin{frame}[fragile]
\frametitle{Overview}
\begin{itemize}

\item Much of what is useful to do in Python is reading files, manipulating the data, and writing out results in another format
\item Python and Numpy provide ways to read and write ASCII and binary files. We will focus on ASCII files
\end{itemize}
\end{frame}

\begin{frame}[fragile]
\frametitle{Background Information}
\url{http://waterservices.usgs.gov/}
\url{http://nwis.waterdata.usgs.gov/nwis/pmcodes/}
\end{frame}

\begin{frame}[fragile]
\frametitle{\texttt{np.genfromtxt}: flexible way to read columns}
Example file: STATE\_FIPS.csv \\
\texttt{State Abbreviation,FIPS Code,State Name} \\
\texttt{AK,02,ALASKA} \\
\texttt{AL,01,ALABAMA}  \\
\pause
\begin{lstlisting}
import numpy as np
infilename = 'STATE_FIPS.csv'
indat = np.genfromtxt(infilename,delimiter=',',dtype=None,names=True)
\end{lstlisting}
\pause
\begin{description}
\item[\texttt{delimiter=','}] delimiter can be \emph{anything}
\item[\texttt{dtype=None}] Numpy interprets column data types. If unknown, makes it a string
\item[\texttt{names=True}] Each column gets a data type and a name
\end{description}
\pause
\begin{lstlisting}
In [10]: indat
Out[7]: 
array([('AK', 2, 'ALASKA'), ('AL', 1, 'ALABAMA'), ('AR', 5, 'ARKANSAS'), ... 
dtype=[('State_Abbreviation', '|S2'), ('FIPS_Code', '<i4'), ('State_Name', '|S20')])
\end{lstlisting}
\pause
\hskip 1.5 cm \emph{N.B.$\rightarrow$underscore replaces space in names}
\end{frame}


\end{document}