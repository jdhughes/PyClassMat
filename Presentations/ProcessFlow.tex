\documentclass[t]{beamer}
% \documentclass[handout,xcolor=pdftex,dvipsnames,table]{beamer}

\mode<presentation>
{
  \usetheme{Warsaw}
  \usecolortheme{whale}
  % or ...

  % or whatever (possibly just delete it)
  \setbeamertemplate{navigation symbols}{}
}

\usepackage{hyperref}

% make a command to decrease font size locally when desired
\newcommand\Fontvi{\fontsize{8.5}{7.2}\selectfont}



\setbeamertemplate{itemize items}[ball]
\setbeamertemplate{itemize subitem}[triangle]
\setbeamertemplate{itemize subsubitem}[circle]
\setbeamercovered{invisible}

\usepackage{palatino} 
\usepackage{listings} % Gives syntax highlighting for python code. 
\usepackage{color} % Used for syntax highlighting. 
\usepackage{textcomp} % Used for syntax highlighting. 
\usepackage{caption}
\captionsetup{labelformat=empty,labelsep=none}
% This gives syntax highlighting in the python environment 
\definecolor{gray}{gray}{0.5} 
\definecolor{key}{rgb}{0,0.5,0} 
\lstset{
language=python,
basicstyle=\ttfamily\tiny, 
otherkeywords={1, 2, 3, 4, 5, 6, 7, 8 ,9 , 0, -, =, +, [, ], (, ), \{, \}, :, *, !}, 
keywordstyle=\color{blue}, 
stringstyle=\color{red},
showstringspaces=false,
alsoletter={1234567890},
otherkeywords={\ , \}, \{},
keywordstyle=\color{blue},
emph={access,and,break,class,continue,def,del,elif ,else,%
except,exec,finally,for,from,global,if,import,in,is,%
lambda,not,or,pass,print,raise,return,try,while},
emphstyle=\color{black}\bfseries,
emph={[2]True, False, None, self},
emphstyle=[2]\color{green},
emph={[3]from, import, as},
emphstyle=[3]\color{blue},
upquote=true,
morecomment=[s]{"""}{"""},
commentstyle=\color{gray}\slshape,
emph={[4]1, 2, 3, 4, 5, 6, 7, 8, 9, 0},
emphstyle=[4]\color{blue},
literate=*{:}{{\textcolor{blue}:}}{1}%
{=}{{\textcolor{blue}=}}{1}%
{-}{{\textcolor{blue}-}}{1}%
{+}{{\textcolor{blue}+}}{1}%
{*}{{\textcolor{blue}*}}{1}%
{!}{{\textcolor{blue}!}}{1}%
{(}{{\textcolor{blue}(}}{1}%
{)}{{\textcolor{blue})}}{1}%
{[}{{\textcolor{blue}[}}{1}%
{]}{{\textcolor{blue}]}}{1}%
{<}{{\textcolor{blue}<}}{1}%
{>}{{\textcolor{blue}>}}{1},%
numbers=none,
}

\newcommand{\putat}[3]{\begin{picture}(0,0)(0,0)\put(#1,#2){#3}\end{picture}}



\title[]{Python Workshop\\
 \includegraphics[scale=0.055]{figures/python-app.png}\hspace{5 pt}Process Flow\hspace{5 pt}\includegraphics[scale=0.055]{figures/python-app.png}}

\author[Hughes] % (optional, use only with lots of authors)
{Joseph D.~Hughes}
\institute[USGS] % (optional, but mostly needed)
{
  U.S. Geological Survey\\
  Florida Water Science Center, Tampa, Florida USA
  }
  \titlegraphic{\includegraphics[scale=0.5]{figures/c_USGSid1.pdf}}
  

\date[UQ12] % (optional, should be abbreviation of conference name)
{USGS National Groundwater Workshop, August 2012}

\subject{Python}


\begin{document}

\begin{frame}
  \titlepage
\end{frame}
\logo{\vspace{-0.3cm} \includegraphics[width=1.5cm]{figures/c_USGSid1.pdf}\hspace*{11.10cm}}  

\begin{frame}{Outline}
\tableofcontents
\end{frame}

\section{Background Information}
\begin{frame}[fragile]
\frametitle{Background Information}
\begin{itemize}
\item Example python script \texttt{ProcessFlowExamples.py}

\item Process flow control resources: \\
\href{http://docs.python.org/tutorial/controlflow.html}{\texttt{\small{\textcolor{blue}{http://docs.python.org/tutorial/controlflow.html}}}}
\end{itemize}
\end{frame}

\section{Process Flow Options}
\subsection{while iterator, continue, and break}
\begin{frame}{\texttt{\textbf{while}}, \texttt{\textbf{continue}}, and \texttt{\textbf{break}}}
\small{Import data from an external file and iterate over data using \texttt{\textbf{while}} and \texttt{\textbf{print}} last entry.}
  \begin{figure}[ht]
  \centering
        \lstset{numbers=left}
        \lstinputlisting[language=python, firstline=1,lastline=18]{python/ProcessFlowExamples.py}
   \end{figure}
\end{frame}

\subsection{range iterator}
\begin{frame}{\texttt{\textbf{range}} iterator and \texttt{\textbf{continue}}}
\small{Import data from an external file and iterate over data using \texttt{\textbf{range}}  and \texttt{\textbf{print}}  last entry.}
  \begin{figure}[ht]
  \centering
        \lstset{numbers=left}
        \lstinputlisting[language=python, firstline=1,lastline=5]{python/ProcessFlowExamples.py}
        \lstinputlisting[language=python, firstline=19,lastline=25,firstnumber=19]{python/ProcessFlowExamples.py}
   \end{figure}
\end{frame}

\subsection{xrange iterator}
\begin{frame}{\texttt{\textbf{xrange}} iterator and \texttt{\textbf{continue}}}
\small{Import data from an external file and iterate over data using \texttt{\textbf{xrange}}  and \texttt{\textbf{print}}  last entry.}
  \begin{figure}[ht]
  \centering
        \lstset{numbers=left}
        \lstinputlisting[language=python, firstline=1,lastline=5]{python/ProcessFlowExamples.py}
        \lstinputlisting[language=python, firstline=26,lastline=32,firstnumber=26]{python/ProcessFlowExamples.py}
   \end{figure}
\end{frame}

\subsection{element iterator}
\begin{frame}{\texttt{\textbf{in}} iterator and \texttt{\textbf{continue}}}
\small{Import data from an external file and iterate over data using \texttt{\textbf{in}} iterator and \texttt{\textbf{print}}  last entry.}
  \begin{figure}[ht]
  \centering
        \lstset{numbers=left}
        \lstinputlisting[language=python, firstline=1,lastline=5]{python/ProcessFlowExamples.py}
        \lstinputlisting[language=python, firstline=33,lastline=42,firstnumber=33]{python/ProcessFlowExamples.py}
   \end{figure}
\end{frame}

\subsection{enumeration iterator}
\begin{frame}{\texttt{enumerate} iterator and \texttt{\textbf{continue}}}
\small{Import data from an external file and iterate over data using \texttt{enumerate} iterator and \texttt{\textbf{print}}  last entry.}
  \begin{figure}[ht]
  \centering
        \lstset{numbers=left}
        \lstinputlisting[language=python, firstline=1,lastline=5]{python/ProcessFlowExamples.py}
        \lstinputlisting[language=python, firstline=43,lastline=48,firstnumber=43]{python/ProcessFlowExamples.py}
   \end{figure}
\end{frame}

\subsection{enumeration iterator with zip}
\begin{frame}{\texttt{enumerate} iterator with \texttt{zip} and \texttt{\textbf{continue}}}
\small{Import data from an external file and iterate over data using \texttt{enumerate} iterator with \texttt{zip} and \texttt{\textbf{print}}  last entry.}
  \begin{figure}[ht]
  \centering
        \lstset{numbers=left}
        \lstinputlisting[language=python, firstline=1,lastline=5]{python/ProcessFlowExamples.py}
        \lstinputlisting[language=python, firstline=49,lastline=55,firstnumber=55]{python/ProcessFlowExamples.py}
   \end{figure}
\end{frame}

\section{Results}
\begin{frame}[fragile]
\frametitle{\texttt{ProcessFlowExamples.py} output}
  \begin{figure}[ht]
  \centering
        \includegraphics[height=0.85\textheight]{figures/ProcessFlowExamplesPythonShellOutput}
   \end{figure}
\end{frame}

\begin{frame}{Run a simple example}
	\begin{columns}
		\column{0.5\textwidth}
			\begin{itemize}
				\item open the command line
				\item type \texttt{python}
				\item enter the text listed below
			\end{itemize}
			\vspace{-15pt}\begin{figure}[ht]
  				\centering
	        		\lstinputlisting[language=python,firstline=1,lastline=5]{python/ProcessFlowInClass.py}
 			\end{figure}
			\vspace{-15pt}\begin{itemize}
				\item then try
			\end{itemize}
			\vspace{-15pt}\begin{figure}[ht]
  				\centering
	        		\lstinputlisting[language=python,firstline=7,lastline=8]{python/ProcessFlowInClass.py}
 			\end{figure}
		\column{0.5\textwidth}
			\vspace{-15pt}\begin{figure}[ht]
				\centering
        			\includegraphics[height=0.95\textwidth]{figures/ProcessFlowInClass.png}
			\end{figure}
	\end{columns}
\end{frame}


\end{document}