
\documentclass{beamer}
% \documentclass[handout,xcolor=pdftex,dvipsnames,table]{beamer}

\mode<presentation>
{
  \usetheme{Warsaw}
  \usecolortheme{whale}
  % or ...

  \setbeamercovered{transparent}
  % or whatever (possibly just delete it)
  \setbeamertemplate{navigation symbols}{}
}

\setbeamertemplate{itemize items}[ball]
\setbeamertemplate{itemize subitem}[triangle]
\setbeamertemplate{itemize subsubitem}[circle]
\setbeamercovered{transparent}

%alternative fonts that provide a nice look
%\usepackage[T1]{fontenc}
%\usepackage{lmodern}

\usepackage{palatino} 
\usepackage{listings} % Gives syntax highlighting for python code. 
\usepackage{color} % Used for syntax highlighting. 
\usepackage{textcomp} % Used for syntax highlighting. 

% This gives syntax highlighting in the python environment 
\definecolor{gray}{gray}{0.5} 
\definecolor{key}{rgb}{0,0.5,0} 
\lstset{
language=python,
basicstyle=\ttfamily\tiny, 
otherkeywords={1, 2, 3, 4, 5, 6, 7, 8 ,9 , 0, -, =, +, [, ], (, ), \{, \}, :, *, !}, 
keywordstyle=\color{blue}, 
stringstyle=\color{red},
showstringspaces=false,
alsoletter={1234567890},
otherkeywords={\ , \}, \{},
keywordstyle=\color{blue},
emph={access,and,break,class,continue,def,del,elif ,else,%
except,exec,finally,for,from,global,if,import,in,is,%
lambda,not,or,pass,print,raise,return,try,while},
emphstyle=\color{black}\bfseries,
emph={[2]True, False, None, self},
emphstyle=[2]\color{green},
emph={[3]from, import, as},
emphstyle=[3]\color{blue},
upquote=true,
morecomment=[s]{"""}{"""},
commentstyle=\color{gray}\slshape,
emph={[4]1, 2, 3, 4, 5, 6, 7, 8, 9, 0},
emphstyle=[4]\color{blue},
literate=*{:}{{\textcolor{blue}:}}{1}%
{=}{{\textcolor{blue}=}}{1}%
{-}{{\textcolor{blue}-}}{1}%
{+}{{\textcolor{blue}+}}{1}%
{*}{{\textcolor{blue}*}}{1}%
{!}{{\textcolor{blue}!}}{1}%
{(}{{\textcolor{blue}(}}{1}%
{)}{{\textcolor{blue})}}{1}%
{[}{{\textcolor{blue}[}}{1}%
{]}{{\textcolor{blue}]}}{1}%
{<}{{\textcolor{blue}<}}{1}%
{>}{{\textcolor{blue}>}}{1},%
numbers=none,
}

\title[]{Python Workshop\\
Built-In Data Structures}

\author[Langevin] % (optional, use only with lots of authors)
{C.D.~Langevin}
\institute[USGS] % (optional, but mostly needed)
{
  U.S. Geological Survey\\
  Reston, Virginia, USA
  }

\date[UQ12] % (optional, should be abbreviation of conference name)
{USGS National Groundwater Workshop, August 2012}

\subject{Python}


\begin{document}

\begin{frame}
  \titlepage
\end{frame}

\begin{frame}{Outline}
\tableofcontents
\end{frame}

\section{Introduction}
\begin{frame}[fragile]{Concepts}
\begin{itemize}
\item{Data Structure -- A way to store and organize data in a computer}
\item{Mutability -- An \emph{immutable} object is one whose property or state cannot be changed, whereas, \emph{mutable} objects can change state}
\end{itemize}
\end{frame}

\begin{frame}[fragile]{Object Terminology}
Everything is an object in Python
\begin{itemize}
\item{Object -- An instance of a data structure}
\item{Function -- Works on an object that is passed into it\\ e.g. {\color{blue}run(model)}}
\item{Method -- A set of instructions that works on itself\\ e.g. {\color{blue}model.run()}}
\item{Member -- A piece of information, such as an integer, that is part of an object\\ e.g. {\color{blue}model.nlay}}
\end{itemize}
\end{frame}


\section{Built-In Data Structures}

\subsection{Numbers}

\begin{frame}[fragile]{Numbers}
\begin{columns}[c]
\column{2.5in}
\begin{itemize}
\item{integer (int)}
\item{float/double precision (float)}
\item{long integer (long)}
\item{complex (complex)}
\end{itemize}

\column{2.5in}
\begin{lstlisting}
In [2]: type(2)
Out[2]: <type 'int'>

In [3]: type(2.0)
Out[3]: <type 'float'>

In [4]: type(2**100)
Out[4]: <type 'long'>

In [5]: type(2j)
Out[5]: <type 'complex'>
\end{lstlisting}
\end{columns}
\end{frame}


\subsection{Strings}

\begin{frame}[fragile]{Strings}
\begin{columns}[c]
\column{2.5in}
\begin{itemize}
\item{type 'str'}
\item{An immutable sequence of characters}
\item{Individual characters are accessed using zero-based indexing}
\end{itemize}

\column{2.5in}
\tiny
\begin{lstlisting}
In [69]: s = 'modflow'

In [70]: s.upper()
Out[70]: 'MODFLOW'

In [71]: s.capitalize()
Out[71]: 'Modflow'

In [72]: s[0]
Out[72]: 'm'

In [73]: s[-1]
Out[73]: 'w'

In [74]: s[0:4]
Out[74]: 'modf'

In [75]: len(s)
Out[75]: 7

In [76]: 'Modflow' + '-88'
Out[76]: 'Modflow-88'
\end{lstlisting}
\end{columns}
\end{frame}


\subsection{Lists}
\begin{frame}[fragile]{Lists}
\begin{columns}[c]
\column{2.5in}
\begin{itemize}
\item{A mutable collection of objects}
\item{List members are accessed using a zero-based indexing scheme}
\item{A list can contain different types}
\end{itemize}

\column{2.5in}
\tiny
\begin{lstlisting}
In [79]: l = []

In [80]: l.append('first')

In [81]: l.append(2)

In [82]: l.append(3.0)

In [83]: l
Out[83]: ['first', 2, 3.0]

In [84]: l[0]
Out[84]: 'first'

In [85]: l[1]
Out[85]: 2

In [86]: l[2]
Out[86]: 3.0

In [87]: len(l)
Out[87]: 3
\end{lstlisting}
\end{columns}
\end{frame}

\begin{frame}[fragile]{List Methods}
\begin{columns}[c]
\column{2.5in}
\begin{itemize}
\item{append}
\item{count}
\item{extend}
\item{index}
\item{insert}
\item{pop}
\item{remove}
\item{reverse}
\item{sort}
\end{itemize}

\column{2.5in}
\begin{lstlisting}
In [115]: l
Out[115]: ['mf.dis', 'mf.bas', 'mf.pcg', 'mf.lpf']

In [116]: l = ['mf.dis', 'mf.bas', 'mf.lpf', 'mf.pcg']

In [117]: l.index('mf.bas')
Out[117]: 1

In [118]: l.remove('mf.pcg')

In [119]: l.append('mf.sip')

In [120]: l
Out[120]: ['mf.dis', 'mf.bas', 'mf.lpf', 'mf.sip']
\end{lstlisting}
\end{columns}
\end{frame}

\subsection{Tuples}
\begin{frame}[fragile]{Tuples}
\begin{columns}[c]
\column{2.5in}
\begin{itemize}
\item{type 'tuple'}
\item{An immutable collection of objects}
\item{Tuple members are accessed using a zero-based indexing scheme}
\item{A tuple can contain different types}
\end{itemize}

\column{2.5in}
\tiny
\begin{lstlisting}
In [1]: point1 = (0, 10, 0)

In [2]: point2 = (10, 0, 0)

In [3]: point3 = (0, 0, 0)

In [4]: triangle = (point1, point2, point3)

In [5]: type(triangle)
Out[5]: <type 'tuple'>

In [6]: triangle[0]
Out[6]: (0, 10, 0)

In [7]: triangle[-1]
Out[7]: (0, 0, 0)

In [8]: point1 = (-1, -1, -1)

In [9]: triangle
Out[9]: ((0, 10, 0), (10, 0, 0), (0, 0, 0))
\end{lstlisting}
\end{columns}
\end{frame}

\subsection{Dictionaries}
\begin{frame}[fragile]{Dictionaries}
\begin{columns}[c]
\column{2.5in}
\begin{itemize}
\item{type 'dict'}
\item{A mutable collection of keys and values where d[key] = value}
\item{Useful for quickly looking up a value based on a key}
\item{Keys must be immutable}
\item{There is no guarantee that the order of the dictionary will be the same as the creation order}
\end{itemize}

\column{2.5in}
\tiny
\begin{lstlisting}
In [21]: d = {}

In [22]: d[1] = 'January'

In [23]: d[2] = 'February'

...

In [33]: d[12] = 'December'

In [34]: d
Out[34]:
{1: 'January',
 2: 'February',
 3: 'March',
 4: 'April',
 5: 'May',
 6: 'June',
 7: 'July',
 8: 'August',
 9: 'September',
 10: 'October',
 11: 'November',
 12: 'December'}
\end{lstlisting}
\end{columns}
\end{frame}


\begin{frame}[fragile]{Accessing Dictionary Keys}
\begin{columns}[c]
\column{2.5in}
\begin{itemize}
\item{A value can be looked up in the dictionary by referencing d[key]}
\end{itemize}

\column{2.5in}
\begin{lstlisting}
In [59]: d[1]
Out[59]: 'January'

In [60]: d[12]
Out[60]: 'December'

In [61]: s = 'It is the month of ' + d[8]

In [62]: s
Out[62]: 'It is the month of August'

In [63]: l = d.keys()

In [64]: l
Out[64]: [1, 2, 3, 4, 5, 6, 7, 8, 9, 10, 11, 12]
\end{lstlisting}
\end{columns}
\end{frame}


\begin{frame}[fragile]{Key Types}
\begin{columns}[c]
\column{2.5in}
\begin{itemize}
\item{Keep in mind that all types cannot be keys}
\end{itemize}

\column{2.5in}
\begin{lstlisting}
In [73]: d = { (1,1,1):'cell 1', (2,1,1):'cell 2' }

In [74]: s = 'this is cell: ' + d[(2,1,1)]

In [75]: s
Out[75]: 'this is cell: cell 2'

In [76]: d = { [1,1,1]:'cell 1', (2,1,1):'cell 2' }
---------------------------------------------------------------------------
TypeError                                 Traceback (most recent call last)

TypeError: unhashable type: 'list'

\end{lstlisting}
\end{columns}
\end{frame}


\begin{frame}[fragile]{Iterating through a Dictionary}
\begin{columns}[c]
\column{2.5in}
\begin{itemize}
\item{It is possible to iterate through all the key:value pairs in a dictionary}
\end{itemize}

\column{2.5in}
\begin{lstlisting}
In [89]: for k, v in d.iteritems():
   ....:     print 'key: ', k
   ....:     print 'value: ',v
   ....:
   ....:
key:  (1, 1, 1)
value:  cell 1
key:  (2, 1, 1)
value:  cell 2
\end{lstlisting}
\end{columns}
\end{frame}



\subsection{Others}
\begin{frame}[fragile]{Others}
Sets\\
Boolean\\
None
\end{frame}

\section{Working with Data Structures}
\subsection{Shared References}
\begin{frame}[fragile]{Shared References}
\begin{itemize}
\item{In-Place Changes}
\item{Shared References and Equality}
\end{itemize}
\tiny{See pages 116-121 in Learning Python, Third Edition}
\end{frame}

\subsection{Zip}
\begin{frame}[fragile]{Zip}
\begin{itemize}
\item{What is Zip?}
\end{itemize}
\end{frame}

\section{Summary}
\begin{frame}[fragile]{Summary}
\end{frame}


\end{document}


